\documentclass{article}
\usepackage{natbib,amssymb}
\begin{document}

\section*{Data reduction protocol}

Two approaches have been used for in-situ U-Th-He dating. In the
first, which we shall call the `first principles method', the molar
concentrations of U, Th, (Sm) and He measured and simply plugged into
the He ingrowth equation to calculate the age t:

\begin{equation}
\begin{array}{rl}
He/V = & \Big[ 8 \frac{137.88}{138.88} (e^{\lambda_{238}t}-1) + 
\frac{7}{138.88} (e^{\lambda_{235}t}-1)\Big] U \\ 
~ & + 6 (e^{\lambda_{232}t}-1) Th + 0.1499 8 (e^{\lambda_{147}t}-1) Sm 
\end{array}
\label{eq:boyce}
\end{equation}

where U, Th and Sm are expressed in moles/$\mu m^3$ or similar units,
He is the molar abundance of helium released from the ablation pit (in
moles) and V is the ablation pit volume (in $\mu m^3$)
\citep{boyce2006}. Equation \ref{eq:boyce} requires normalisation of
the U, Th and Sm concentrations to a concentration standard such as
NIST SRM610 or 612 glass, and the calculation of He concentrations by
dividing the molar helium abundance by the ablation pit volume. Though
simple in principle, these measurements are often difficult in
practice. While the precision of U, Th, Sm concentration and ablation
volume measurements may be very precise, their accuracy often leaves
much to be desired. The reason for this are the different ablation
characteristics of glass and zircon, and the redeposition of ejecta in
ablation pits under ultra-high vacuum conditions. As a result, the
first principles method often yields ages that are offset from the
true values by an unquantifiable systematic error. An example of this
problem is given in the results section of this paper (?)\\

To circumvent these problems, \citet{vermeesch2012a}, proposed an
alternative `pairwise dating' approach, in which all the mass
spectrometer and pit volume (or depth) measurements are normalised to
a standard of known U-Th-He age. Thus, all the aforementioned
systematic errors should cancel out, producing more accurate ages.  By
using an age standard that is concordant in its $^{208}$Pb/$^{232}$Th-
and $^{206}$Pb/$^{238}$U-ages, the pairwise dating method removes the
need to use any concentration standards. In the present study, we
introduce a slightly modified version of the pairwise dating method,
which differs from the original implementation in the following ways:

\begin{enumerate}
\item Whereas the original method combines the samples and standards
  on a one-by-one basis (hence the name `pairwise dating'), we here
  combine several standard measurements together in a single
  block. This is possible because the Resochron is equipped with a
  $^3$He spike tank, making it immune to the sensitivity drift which
  was a concern in the general purpose noble gas mass spectrometer
  used by \citet{vermeesch2012a}.
\item Whereas the calculations in the original method were performed
  on the raw data files, the modified method uses the processed
  elemental concentrations as input. This better fits the natural
  workflow of the Resochron, which aims to determine trace elemental
  compositions as well as U-Th-He ages, a process for which it is
  impossible to avoid NIST glass.
\end{enumerate}

As explained at the beginning of this section, both the U, Th and Sm
concentrations and the ablation pit volume measurements often tend to
be inaccurate. All the corresponding systematic errors can be grouped
into a single calibration factor which we shall call $\kappa$:

\begin{equation}
\begin{array}{rcl}
\kappa  & =  (He/V) \Big/ & \Big[ \left( 8 \frac{137.88}{138.88} (e^{\lambda_{238}t}-1) + 
\frac{7}{138.88} (e^{\lambda_{235}t}-1)\right) U \\ 
~& ~ & + 6 (e^{\lambda_{232}t}-1) Th + 0.1499 8 (e^{\lambda_{147}t}-1) Sm \Big]
\end{array}
\label{eq:kappa}
\end{equation}

$\kappa$ is unknown but can be estimated by analysing a standard of
known U-Th-He age t$_s \pm \sigma(t_s)$. Suppose that we have n
standard measurements, and assume that the corresponding age estimates
$\hat{t}_s^i$ follow a Normal distribution with two sources of
variance:

\begin{equation}
\hat{t}_s^i(U_i,Th_i,Sm_i,He_i,V_i,\kappa) \sim N\left(t_s,\sigma^2(t_s) + \sigma^2(\hat{t}_s^i|\kappa)\right)\\
\label{eq:thats}
\end{equation}

with $1 \leq i \leq n$ and

\begin{equation}
\begin{array}{rl}
\sigma^2(\hat{t}_s^i|\kappa) = & \left[\frac{\partial \hat{t}_s^i}{\partial U_i}\right]^2 \sigma_{U_i}^2 +
\left[\frac{\partial \hat{t}_s^i}{\partial Th_i}\right]^2 \sigma_{Th_i}^2 +
\left[\frac{\partial \hat{t}_s^i}{\partial Sm_i}\right]^2 \sigma_{Sm_i}^2 + \\
~ & \left[\frac{\partial \hat{t}_s^i}{\partial He_i}\right]^2 \sigma_{He_i}^2 +
\left[\frac{\partial \hat{t}_s^i}{\partial V_i}\right]^2 \sigma_{V_i}^2
\end{array}
\label{eq:sigmai}
\end{equation}

Then $\kappa$ can be found by maximising the log-likelihood function:

\begin{equation}
\mathcal{L} \propto - \sum\limits_{i=1}^{n} \left[
\frac{\left(t_s - \hat{t}_s^i(\kappa)\right)^2}{\sigma^2(t_s) + \sigma^2(\hat{t}_s^i|\kappa)} 
+ \frac{ln\left(\sigma^2(t_s) + \sigma^2(\hat{t}_s^i|\kappa)\right)}{2} \right]
\label{eq:L}
\end{equation}

This function can be solved in a just a few iterations with Newton's
method, which involves taking the first and second derivatives of
$\mathcal{L}$ with respect to $\kappa$.  This is convenient because
the latter can then be used to estimate the approximate standard error
of $\kappa$:

\begin{equation}
\sigma^2(\kappa) \approx -\frac{1}{\partial^2\mathcal{L}/\partial\kappa^2}
\label{eq:fisher}
\end{equation}

using standard maximum likelihood theory.  The resulting
$\kappa$-value can then simply be plugged into Equation \ref{eq:kappa}
and solved for $\hat{t}_x^i$. The age uncertainty is given by:

\begin{equation}
\begin{array}{rl}
\sigma^2(\hat{t}_x^i) = 
& \left[\frac{\partial \hat{t}_x^i}{\partial U_i}\right]^2 \sigma_{U_i}^2 +
\left[\frac{\partial \hat{t}_x^i}{\partial Th_i}\right]^2 \sigma_{Th_i}^2 +
\left[\frac{\partial \hat{t}_x^i}{\partial Sm_i}\right]^2 \sigma_{Sm_i}^2 +\\
~ & \left[\frac{\partial \hat{t}_x^i}{\partial He_i}\right]^2 \sigma_{He_i}^2 +
\left[\frac{\partial \hat{t}_x^i}{\partial V_i}\right]^2 \sigma_{V_i}^2 +
\left[\frac{\partial \hat{t}_x^i}{\partial \kappa}\right]^2 \sigma_{\kappa}^2
\end{array}
\label{eq:sigmatxi}
\end{equation}

which accounts for all sources of uncertainty, including those on
$\kappa$ and t$_s$.\\

Although the above calculations are relatively straightforward to
carry out, the details of taking the partial derivatives are rather
tedious. We have implemented the method in a user-friendly
browser-based calculator to facilitate the application of the
$\kappa$-calibration method. The spreadsheet-like app is entirely
written in HTML and JavaScript and can therefore be downloaded and run
offline as well as online. The calculator is available free of charge
at \\{\tt http://resochronometer.london-geochron.com}.

\bibliographystyle{plainnat}
\bibliography{/home/pvermees/Dropbox/biblio}

\end{document}
